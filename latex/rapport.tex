\documentclass[12pt,a4paper]{article}

% === Packages ===
\usepackage[utf8]{inputenc}
\usepackage[T1]{fontenc}
\usepackage[french]{babel}
\usepackage{geometry}
\usepackage{graphicx}
\usepackage{listings}
\usepackage{xcolor}
\usepackage{hyperref}
\usepackage{amsmath}
\usepackage{booktabs}
\usepackage{float}
\usepackage{enumitem}

% === Configuration de la page ===
\geometry{margin=2.5cm}

% === Configuration des liens ===
\hypersetup{
    colorlinks=true,
    linkcolor=blue,
    urlcolor=blue
}

% === Configuration du code source ===
\definecolor{codegreen}{rgb}{0,0.6,0}
\definecolor{codegray}{rgb}{0.5,0.5,0.5}
\definecolor{backcolour}{rgb}{0.95,0.95,0.92}

\lstdefinestyle{mystyle}{
    backgroundcolor=\color{backcolour},
    commentstyle=\color{codegreen},
    keywordstyle=\color{magenta},
    basicstyle=\ttfamily\small,
    breaklines=true,
    frame=single,
    tabsize=2
}
\lstset{style=mystyle}

% === Début du document ===
\begin{document}

% === Titre ===
\begin{center}
    {\LARGE\textbf{Documentation du Projet}}\\[0.5cm]
    {\Large Détection de Communautés dans un Réseau d'Amis}\\[1cm]
\end{center}

% ============================================================================
\section{Présentation du Projet}
% ============================================================================

\subsection{Objectif Principal}
Identifier des \textbf{sous-groupes cohérents} (communautés) dans un réseau social de 20-30 utilisateurs. Une communauté est un groupe d'utilisateurs plus connectés entre eux qu'avec le reste du réseau.

\subsection{Ce que nous allons faire}
\begin{enumerate}
    \item Créer un fichier de données avec les relations d'amitié (CSV)
    \item Construire un graphe à partir de ces données
    \item Appliquer 2 algorithmes de détection de communautés
    \item Visualiser les résultats graphiquement
    \item Comparer les algorithmes
    \item Analyser les relations internes et externes
\end{enumerate}

% ============================================================================
\section{Structure des Dossiers}
% ============================================================================

\begin{lstlisting}[language=bash]
Projet-Structure-15/
|
|-- data/
|   +-- reseau_amis.csv       # Les donnees (relations d'amitie)
|
|-- src/
|   |-- graphe.py             # Construire le graphe
|   |-- louvain.py            # Algorithme 1 : Louvain
|   |-- girvan_newman.py      # Algorithme 2 : Girvan-Newman
|   |-- comparaison.py        # Comparer les 2 algorithmes
|   |-- visualisation.py      # Dessiner les graphes
|   +-- analyse.py            # Analyser les communautes
|
|-- resultats/                # Images generees
|-- latex/                    # Ce document
|-- main.py                   # Programme principal
+-- requirements.txt          # Bibliotheques a installer
\end{lstlisting}

% ============================================================================
\section{Description de Chaque Fichier}
% ============================================================================

% --- DATA ---
\subsection{data/reseau\_amis.csv}

\textbf{Rôle :} Contient les données du réseau social.

\textbf{Format :}
\begin{lstlisting}
utilisateur1,utilisateur2
Alice,Bob
Alice,Charlie
Bob,Charlie
\end{lstlisting}

\textbf{Contenu actuel :}
\begin{itemize}
    \item 20 utilisateurs (nœuds)
    \item 5 groupes naturels de 4 personnes chacun
    \item Des connexions entre les groupes (ponts)
\end{itemize}

% --- GRAPHE.PY ---
\subsection{src/graphe.py}

\textbf{Rôle :} Charger les données et construire le graphe.

\textbf{Fonctions :}
\begin{itemize}
    \item \texttt{charger\_donnees(chemin)} -- Lit le fichier CSV
    \item \texttt{construire\_graphe(df)} -- Crée le graphe NetworkX
    \item \texttt{afficher\_informations\_graphe(G)} -- Affiche les stats (nœuds, arêtes, densité...)
    \item \texttt{charger\_graphe\_complet(chemin)} -- Fait tout en une seule étape
\end{itemize}

\textbf{Bibliothèques utilisées :} pandas, networkx

% --- LOUVAIN.PY ---
\subsection{src/louvain.py}

\textbf{Rôle :} Détecter les communautés avec l'algorithme de Louvain.

\textbf{Fonctions prévues :}
\begin{itemize}
    \item \texttt{detecter\_communautes(G)} -- Applique Louvain et retourne les groupes
    \item \texttt{calculer\_modularite(G, partition)} -- Calcule la qualité de la partition
    \item \texttt{afficher\_communautes(partition)} -- Affiche les membres de chaque groupe
\end{itemize}

\textbf{Bibliothèque utilisée :} python-louvain (community)

% --- GIRVAN-NEWMAN.PY ---
\subsection{src/girvan\_newman.py}

\textbf{Rôle :} Détecter les communautés avec l'algorithme de Girvan-Newman.

\textbf{Fonctions prévues :}
\begin{itemize}
    \item \texttt{detecter\_communautes(G, k)} -- Applique Girvan-Newman pour k communautés
    \item \texttt{calculer\_modularite(G, communautes)} -- Calcule la qualité
    \item \texttt{afficher\_communautes(communautes)} -- Affiche les groupes
\end{itemize}

\textbf{Bibliothèque utilisée :} networkx (intégré)

% --- COMPARAISON.PY ---
\subsection{src/comparaison.py}

\textbf{Rôle :} Comparer les résultats des deux algorithmes.

\textbf{Fonctions prévues :}
\begin{itemize}
    \item \texttt{comparer\_algorithmes(G)} -- Exécute les 2 algos et compare
    \item \texttt{afficher\_tableau\_comparaison(resultats)} -- Affiche un tableau récapitulatif
    \item \texttt{mesurer\_temps\_execution(algo, G)} -- Mesure le temps de chaque algo
\end{itemize}

\textbf{Métriques comparées :}
\begin{itemize}
    \item Modularité (qualité de la partition)
    \item Temps d'exécution
    \item Nombre de communautés détectées
    \item Taille moyenne des communautés
\end{itemize}

% --- VISUALISATION.PY ---
\subsection{src/visualisation.py}

\textbf{Rôle :} Créer les visualisations graphiques.

\textbf{Fonctions prévues :}
\begin{itemize}
    \item \texttt{dessiner\_graphe\_simple(G)} -- Dessine le graphe sans couleurs
    \item \texttt{dessiner\_communautes(G, partition, titre)} -- Dessine avec couleurs par communauté
    \item \texttt{dessiner\_comparaison(G, partition1, partition2)} -- Côte à côte
    \item \texttt{sauvegarder\_image(fig, nom)} -- Sauvegarde dans resultats/
\end{itemize}

\textbf{Bibliothèque utilisée :} matplotlib

% --- ANALYSE.PY ---
\subsection{src/analyse.py}

\textbf{Rôle :} Analyser les communautés détectées.

\textbf{Fonctions prévues :}
\begin{itemize}
    \item \texttt{compter\_aretes\_internes(G, communaute)} -- Arêtes dans le groupe
    \item \texttt{compter\_aretes\_externes(G, communaute)} -- Arêtes vers d'autres groupes
    \item \texttt{calculer\_densite(G, communaute)} -- Densité du groupe
    \item \texttt{analyser\_toutes\_communautes(G, partition)} -- Analyse complète
    \item \texttt{afficher\_rapport(analyse)} -- Affiche les résultats
\end{itemize}

\textbf{Métriques calculées :}
\begin{itemize}
    \item Arêtes internes vs externes
    \item Ratio interne/externe
    \item Densité de chaque communauté
\end{itemize}

% --- MAIN.PY ---
\subsection{main.py}

\textbf{Rôle :} Point d'entrée qui exécute tout le programme.

\textbf{Étapes :}
\begin{enumerate}
    \item Charger le graphe depuis le CSV
    \item Afficher les informations du graphe
    \item Appliquer Louvain
    \item Appliquer Girvan-Newman
    \item Comparer les deux algorithmes
    \item Visualiser les résultats
    \item Analyser les communautés
    \item Sauvegarder les images
\end{enumerate}

% ============================================================================
\section{Les Algorithmes}
% ============================================================================

\subsection{Algorithme de Louvain}

\textbf{Approche :} Bottom-up (agglomérative)

\textbf{Principe :}
\begin{enumerate}
    \item Au début : chaque personne = sa propre communauté
    \item On essaie de déplacer chaque personne vers une communauté voisine
    \item Si ça améliore la modularité, on fait le déplacement
    \item On répète jusqu'à ce qu'on ne puisse plus améliorer
    \item On fusionne les communautés en "super-nœuds" et on recommence
\end{enumerate}

\textbf{Avantages :}
\begin{itemize}
    \item Très rapide
    \item Bons résultats
    \item Trouve automatiquement le nombre de communautés
\end{itemize}

\textbf{Inconvénients :}
\begin{itemize}
    \item Peut manquer de petites communautés
\end{itemize}

\subsection{Algorithme de Girvan-Newman}

\textbf{Approche :} Top-down (divisive)

\textbf{Principe :}
\begin{enumerate}
    \item Au début : tout le monde dans une seule grande communauté
    \item On calcule l'importance de chaque arête (betweenness)
    \item On supprime l'arête la plus importante (le "pont")
    \item On répète jusqu'à avoir le nombre de communautés voulu
\end{enumerate}

\textbf{Avantages :}
\begin{itemize}
    \item Détecte bien les ponts entre groupes
    \item On contrôle le nombre de communautés
\end{itemize}

\textbf{Inconvénients :}
\begin{itemize}
    \item Plus lent que Louvain
    \item Il faut choisir le nombre de communautés
\end{itemize}

% ============================================================================
\section{La Modularité}
% ============================================================================

La \textbf{modularité} mesure la qualité d'une partition en communautés.

\textbf{Formule :}
$$Q = \frac{1}{2m} \sum_{ij} \left[ A_{ij} - \frac{k_i k_j}{2m} \right] \delta(c_i, c_j)$$

\textbf{Interprétation :}
\begin{itemize}
    \item $Q$ proche de 1 = excellente séparation en communautés
    \item $Q$ proche de 0 = partition aléatoire
    \item $Q$ négatif = pire qu'aléatoire
\end{itemize}

En pratique, une modularité $> 0.3$ indique une bonne structure de communautés.

% ============================================================================
\section{Analyse des Relations}
% ============================================================================

\subsection{Relations Internes}
Connexions entre membres de la \textbf{même} communauté.
$$\text{Arêtes internes} = \text{connexions dans le groupe}$$

\subsection{Relations Externes}
Connexions entre membres de communautés \textbf{différentes}.
$$\text{Arêtes externes} = \text{connexions vers d'autres groupes}$$

\subsection{Densité Interne}
Mesure à quel point un groupe est connecté.
$$\text{Densité} = \frac{\text{arêtes existantes}}{\text{arêtes possibles}} = \frac{2 \times \text{arêtes}}{n \times (n-1)}$$

Une bonne communauté a :
\begin{itemize}
    \item Beaucoup d'arêtes internes
    \item Peu d'arêtes externes
    \item Densité interne élevée
\end{itemize}

% ============================================================================
\section{Comparaison Prévue}
% ============================================================================

\begin{table}[H]
    \centering
    \begin{tabular}{lcc}
        \toprule
        \textbf{Critère} & \textbf{Louvain} & \textbf{Girvan-Newman} \\
        \midrule
        Approche & Bottom-up & Top-down \\
        Vitesse & Rapide & Lent \\
        Nb communautés & Automatique & À choisir \\
        Qualité & Très bonne & Bonne \\
        \bottomrule
    \end{tabular}
\end{table}

% ============================================================================
\section{Comment Exécuter le Projet}
% ============================================================================

\subsection{Installation}
\begin{lstlisting}[language=bash]
# Creer un environnement virtuel
python -m venv .venv
source .venv/bin/activate

# Installer les dependances
pip install -r requirements.txt
\end{lstlisting}

\subsection{Exécution}
\begin{lstlisting}[language=bash]
python main.py
\end{lstlisting}

\subsection{Dépendances (requirements.txt)}
\begin{lstlisting}
networkx>=3.0
pandas>=2.0
matplotlib>=3.7
python-louvain>=0.16
\end{lstlisting}

% ============================================================================
\section{Résultats Attendus}
% ============================================================================

Le programme va générer :

\begin{enumerate}
    \item \textbf{Affichage console :}
    \begin{itemize}
        \item Informations du graphe (nœuds, arêtes, densité)
        \item Communautés détectées par chaque algorithme
        \item Tableau de comparaison
        \item Analyse des relations internes/externes
    \end{itemize}
    
    \item \textbf{Images dans resultats/ :}
    \begin{itemize}
        \item graphe\_original.png -- Le réseau sans couleurs
        \item graphe\_louvain.png -- Communautés Louvain
        \item graphe\_girvan\_newman.png -- Communautés Girvan-Newman
        \item comparaison.png -- Les deux côte à côte
    \end{itemize}
\end{enumerate}

\end{document}
